\section{Data Types}
\label{sec:Data-Types}
%~\cref{sec:Data-Types}
Datatypes range of values can vary from language to language. 
\begin{table}[!h]
\centering
\begin{tabular}{llllll}
               &    & Data Type            &          &                    &   \\
"highest type" & 01 & long double          &          &                    &   \\
               & 02 & double               &          &                    &   \\
               & 03 & float                &          &                    &   \\
               & 04 & unsigned long long int   & $\equiv$ & unsigned long long \st{int}&   \\
               & 05 & long long int        & $\equiv$ & long long \st{int} &   \\
               & 06 & unsigned long int    & $\equiv$ & unsigned long \st{int} &   \\
               & 07 & long int             & $\equiv$ & long \st{int}      &   \\
               & 08 & unsigned int         & $\equiv$ & unsigned \st{int}  &   \\
               & 09 & int                  &          &                    &   \\
               & 10 & unsigned short int   & $\equiv$ & unsigned short \st{int} &   \\
               & 11 & short int            & $\equiv$ & short \st{int}     &   \\
               & 12 & unsigned char        &          &                    &   \\
               & 13 & char and signed char &          &                    &   \\
"lowest type"  & 14 & bool                 &          &                    &  
\end{tabular}
\caption{Data Types}
\label{tab:t_00_Data-types_Cpp}
%~\ref{tab:t_00_Data-types_Cpp}
\end{table}

\subsection{\texttt{size\_t}}
%label
According to the C++ standard \texttt{size\_t} represents an unsigned integral type. It is defined in the \texttt{std} namespace and is in header \texttt{<cstdef>}, which is included by various other headers.

\begin{itemize}
    \item This type is recommended for any variable that represents an array's subscripts.
\end{itemize}

\subsection{The \texttt{bool} value}
Please remember that\\

\begin{table}[!h]
\centering
\begin{tabular}{lll}
\multicolumn{3}{l}{bool behaviour} \\ 
\hline
true & $:=$ & nonzero value \\
false & $:=$ & 0
\end{tabular}
\end{table}
%label