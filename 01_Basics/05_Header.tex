\section{Including the class header in the source code file}
Example of using header (\texttt{.h}) and and the \texttt{.cpp} file\\
\begin{itemize}
    \item Inside the header we will define the class declaration.
    \item Mainwhile in the \texttt{.cpp} we will define the class-member functions.
\end{itemize}

\subsection{Header \texttt{.h}}
\begin{minipage}{.5\textwidth}
\end{minipage}\hspace{0.0cm}
\begin{minipage}{.8\textwidth}
\vspace{0.1cm}
\begin{lstlisting}[frame=tlrb,numbers=none,mathescape=true,escapechar=\%,columns=flexible]
#include <string.h>

//prevent multiple inclussion of header
#ifndef DATE_H
#define DATE_H

class Date{
public:
    explicit Date(unsigned int = 1, unsigned int = 1, unsigned int = 2000)
    std::string toString() const;
    
private:
    unsigned int moth;
    unsigned int day;
    unsigned int year;
}; //do not forget the semicolon

#endif
\end{lstlisting}
\end{minipage}

\subsection{\texttt{.cpp}}
\begin{minipage}{.5\textwidth}
\end{minipage}\hspace{0.0cm}
\begin{minipage}{.8\textwidth}
\vspace{0.1cm}
\begin{lstlisting}[frame=tlrb,numbers=none,mathescape=true,escapechar=\%,columns=flexible]
#include <sstream>
#include <string>
#include <Date.h> //INCLUDE DEFINITION OF THE HEADER!
usingnamespace std;

//Date constructor (should do range checking)
Date::Date(unsigned int m, unsigned d unsigned int y)
    : month{m}, day{d}, year{y} {}
    
//print Date in the format mm/dd/yyyy
string Date::toString() const {
    ostringstream output;
    output <<%\,% month <<%\,% '/' <<%\,% day <<%\,% '/' <<%\,% year;
    return output.str();
}
#endif
\end{lstlisting}
\end{minipage}


\subsection{Example}
\begin{minipage}{.5\textwidth}
\end{minipage}\hspace{0.0cm}
\begin{minipage}{.8\textwidth}
\vspace{0.1cm}
\begin{lstlisting}[frame=tlrb,numbers=none,mathescape=true,escapechar=\%,columns=flexible]
#include <iostream>
#include "Date.h" // include definition of class Date from Date.h
using namespace std;

int main(){
    Date date1{7, 4, 2004};
    Date date2; // date2 defaults to 1/1/2000
    
    cout << "date1 = " << date1.toString() << "\ndate2 = " << date2.toString() << "\n\n";
    
    cout << "After default memberwise assignment, date2 = " << date2.toString() << endl;
}
\end{lstlisting}
\end{minipage}

\begin{minipage}{.5\textwidth}
\end{minipage}\hspace{0.0cm}
\begin{minipage}{.8\textwidth}
Output:\\
\vspace{-0.5cm}
\begin{lstlisting}[frame=tlrb,numbers=none,mathescape=true,escapechar=\%,columns=flexible]
date1 = 7/4/2004
date2 = 1/1/2000
After default memberwise assignment, date2 = 7/4/2004
\end{lstlisting}
\end{minipage}