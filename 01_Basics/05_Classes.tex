\section{Classes}

\subsection{Including the class header (*.h) in the source code file (*.cpp)}
Example of using header (\texttt{.h}) and and the \texttt{.cpp} file\\
\begin{itemize}
    \item Inside the header we will define the class declaration.
    \item Mainwhile in the \texttt{.cpp} we will define the class-member functions.
\end{itemize}

\subsection{Header \texttt{.h}}
\begin{minipage}{\MPWxLARGExLISTING\textwidth} % = 0.8 times \textwidth
\vspaceTextToListing % =\vspace{0.1cm}
\begin{CPPCode}
#include <string.h>

//prevent multiple inclussion of header
#ifndef DATE_H
#define DATE_H

class Date{
public:
    explicit Date(unsigned int = 1, unsigned int = 1, unsigned int = 2000)
    std::string toString() const;
    
private:
    unsigned int moth;
    unsigned int day;
    unsigned int year;
}; //¿$\leftarrow$ \codecomment{do not}¿ forget the semicolon

#endif
\end{CPPCode}
\end{minipage}

\subsection{\texttt{.cpp}}
\begin{minipage}{\MPWxLARGExLISTING\textwidth} % = 0.8 times \textwidth
\vspaceTextToListing % =\vspace{0.1cm}
\begin{CPPCode}
#include <¿s¿tring> // ¿\codecomment{for ostringstream class}¿
#include <sstream>
#include <Date.h>   //INCLUDE DEFINITION OF THE HEADER!
using namespace std;

//Date constructor (should do range checking)
Date::Date(unsigned int m, unsigned d unsigned int y)
    : month{m}, day{d}, year{y} {}
    
//print Date in the format mm/dd/yyyy
string Date::toString() const {
    ostringstream output;
    output ¿<¿¿<¿ month ¿<¿¿<¿ '/' ¿<¿¿<¿ day ¿<¿¿<¿ '/' ¿<¿¿<¿ year;
    return output.str();
}
#endif
\end{CPPCode}
\end{minipage}

\newpage
\subsubsection{Example}
\begin{minipage}{\MPWxLARGExLISTING\textwidth} % = 0.8 times \textwidth
\vspaceTextToListing % =\vspace{0.1cm}
\begin{CPPCode}
#include <iostream>
#include "Date.h" // include definition of ¿\codecomment{class}¿ Date from Date.h
using namespace std;

int main()
{
    Date date1{7, 4, 2004};
    Date date2; // date2 defaults to 1/1/2000
    
    cout ¿<¿¿<¿ "date1 = " ¿<¿¿<¿ date1.toString() ¿<¿¿<¿ "\ndate2 = " ¿<¿¿<¿ date2.toString() ¿<¿¿<¿ "\n\n";
    
    cout ¿<¿¿<¿ "After default memberwise assignment, date2 = " ¿<¿¿<¿ date2.toString() ¿<¿¿<¿ endl;
}
\end{CPPCode}
\end{minipage}\\

\noindent{}
\begin{minipage}{\MPWxLARGExLISTING\textwidth} % = 0.8 times \textwidth
\vspaceTextToListing % =\vspace{0.1cm}
Output:\\
\vspace{-0.5cm}
\begin{Terminal}
date1 = 7/4/2004
date2 = 1/1/2000
After default memberwise assignment, date2 = 7/4/2004
\end{Terminal}
\end{minipage}

%%%%%%%%%%%%%%%%%%%%%%%%%%%%%%%
%%%%%%%%% SUB SECTION %%%%%%%%%
%%%%%%%%%%%%%%%%%%%%%%%%%%%%%%%
\subsection{Accessing Class Members}
What elements belong to tha class's scope?
\begin{itemize}
    \item data members
    \item member functions
    \item Remark: nommember functions belong to \texttt{global namespace scope} see Chapter XX.
\end{itemize}


% \usepackage{color}
% \usepackage{tabularray}

\begin{table}[!h]
\centering
\begin{tblr}{
  row{1} = {c},
  cell{1}{1} = {c=3}{},
  cell{2}{2} = {c=2}{},
  cell{3}{1} = {r=3}{},
  cell{3}{2} = {r=3}{},
  vline{1,2}   = {1-6}{},
  vline{3} = {1-5}{},
  vline{1-3} = {2}{},
  vline{-} = {3}{},
  vline{4} = {1-5}{},
  hline{1-3,6} = {-}{},
  hline{7} = {1}{},
  measure = vbox, %needed for itemize
}
Class &  & \\
Within Class´s Scope & Outside Class´s Scope & \\
{\begin{itemize}
    \item Class member are inmediately accesible\\by all of its functions.
    \item Class member are referenced by its name. 
\end{itemize}}
 & public   & referenced through\\
 &  &  \begin{itemize}
            \item an object name,
            \item a reference to an object, or
            \item a pointer to an object
        \end{itemize} \\
 &  & {above options are known as\\\textbf{handles on an object.}}\\
\end{tblr}
\end{table}

\noindent Lets say we have a class Account then we can write:\\
\begin{minipage}{\MPWxLARGExLISTING\textwidth} % = 0.8 times \textwidth
\vspaceTextToListing % =\vspace{0.1cm}
\begin{CPPCode}
Account     account; ¿\hspace{2.2cm}¿// declaration of object
Account& accountRef{account}; ¿\hspace{0.17cm}¿// refers to an object type Account
Account* ptrAccount{&account}; // points to an account object
\end{CPPCode}
\end{minipage}\\

\noindent Now supposse Account has a member function called \CppCommonCode{SetBalance}, if we want to invoke it, then\\
\begin{minipage}{\MPWxLARGExLISTING\textwidth} % = 0.8 times \textwidth
\vspaceTextToListing % =\vspace{0.1cm}
\begin{CPPCode}
float someNumber= 123.45;
account.SetBalance(someNumber); ¿\hspace{0.55cm}¿ // invoke via the object
accountRef.SetBalance(someNumber);¿\hspace{0.37cm}¿// invoke via a reference to the obj.
ptrAccount->SetBalance(someNumber); // invoke via a pointer to the object
\end{CPPCode}
\end{minipage}\\

%%%%%%%%%%%%%%%%%%%%%%%%%%%%%%%
%%%%%%%%% SUB SECTION %%%%%%%%%
%%%%%%%%%%%%%%%%%%%%%%%%%%%%%%%
\subsection{Overload Constructor}
To overload a constructor, \textbf{provide} in the class definition \textbf{a prototype for each version of the constructor}, and provide a separate constructor definition for each overloaded version.

%%%%%%%%%%%%%%%%%%%%%%%%%%%%%%%
%%%%%%%%% SUB SECTION %%%%%%%%%
%%%%%%%%%%%%%%%%%%%%%%%%%%%%%%%
\subsection{Delegating Constructor}
Suppose we have the following overloaded constructor's prototypes\\
\begin{minipage}{\MPWxLARGExLISTING\textwidth} % = 0.8 times \textwidth
\vspaceTextToListing % =\vspace{0.1cm}
\begin{CPPCode}
Time(); ¿\hspace{2.15cm}¿// default hour, min and sec to 0
explicit Time(int); ¿\hspace{0.15cm}¿// default min and sec to 0
Time(int¿¿, int);¿\hspace{0.92cm}¿// default sec to 0
Time(int¿¿, int¿¿, int);
\end{CPPCode}
\end{minipage}\\

\noindent The delegating constructor is a constructor which call to another constructor, i.e.:\\
\begin{minipage}{\MPWxLARGExLISTING\textwidth} % = 0.8 times \textwidth
\vspaceTextToListing % =\vspace{0.1cm}
\begin{CPPCode}
Time::Time() : Time{0,0,0} {} 
Time::Time(int hour) : Time{hour,0,0} {}
Time::Time(int hour, int minute) : Time{hour,minute,0} {}
Time::Time(int hour, int minute, int second) 
{
    setTime(hour, minute, second); // this is just an example.
}
\end{CPPCode}
\end{minipage}\\

%%%%%%%%%%%%%%%%%%%%%%%%%%%%%%%
%%%%%%%%% SUB SECTION %%%%%%%%%
%%%%%%%%%%%%%%%%%%%%%%%%%%%%%%%
\subsection{Destructors}
Is a special member function. To declare a destructor we have to use the tilde ($\sim$).
\begin{itemize}
    \item We know that the tilde operator is the bitwise complement operator, and, in a sense, the destructor is the complement of the constructor.
\end{itemize}

\subsubsection{Con- and Destructor for Objects in Global Scope}
\begin{itemize}
    \item Constructors are called for the objects defined in G.S. \underline{before} any other function in that program
    \item Constructors are called even before \texttt{main()}.
    \item Destructors are called when \texttt{main()} terminates.
    \item If \CppCommonCode{exit} or \CppCommonCode{abort} is performed then Destructors are \underline{not} called.
\end{itemize}

\subsubsection{Con- and Destructor for (Non-static) local objects}
\begin{itemize}
    \item Constructors are called when execution reaches the point where that object is defined.
    \item Destructors are called when execution leaves the object's scope.
    \item I.e. constructor and destructor are called when the object execution enters and leaves the object's scope.
    \item If \CppCommonCode{exit} or \CppCommonCode{abort} is performed then Destructors are \underline{not} called.
\end{itemize}

\subsubsection{Con- and Destructor for static local objects}
\begin{itemize}
    \item Constructors is called only once, when execution first reaches the point where the object is defined.
    \item Destructors are called when \texttt{main()} terminates or \CppCommonCode{exit} is called.
    \item If \CppCommonCode{abort} is performed then destructors are \underline{not} called.
    \item Global and static objects are destroyed in the reverse order of their creation
\end{itemize}

%\begin{itemize}
%            \item an object name,
%            \item a reference to an object, or
%            \item a pointer to an object
%        \end{itemize}