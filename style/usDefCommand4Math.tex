% Math commands
% usDef = user Defined
\makeatletter
\newtheoremstyle{myplain}%
{\item[\hskip\labelsep \theorem@headerfont ##1\ \textup{\bfseries(##2)}\ \,\theorem@separator]}%
{\item[\hskip\labelsep \theorem@headerfont ##1\ \textup{\bfseries(##2)}\ (##3)\,\theorem@separator]}
\makeatother
\theoremstyle{myplain}
\theoremheaderfont{\itshape}
\theorembodyfont{\itshape}
\theoremseparator{\textemdash}
\newtheorem{definition}{Definition}[section]
\newtheorem{satz}{Satz}[section]
\newcommand{\vspaceBemerkung}{\vspace{0.2cm}}
\newcommand{\entspricht}{{\,\hat{=}\,}}
\renewcommand{\qed}{\hfill\blacksquare}
\newcommand{\qedwhite}{\hfill \ensuremath{\Box}}
\newcommand{\mbb}[1]{\mathbb{R}^{#1}}
\newcommand{\mbbb}[2]{\mathbb{R}^{#1\times{#2}}}
\newcommand*\conj[1]{\bar{#1}}
\newcommand*\mean[1]{\bar{#1}}
\newcommand\widebar[1]{\mathop{\overline{#1}}} 

\DeclareMathOperator{\Lagr}{\mathcal{L}}
\DeclareMathOperator{\Lapl}{\mathcal{L}}

%new command for apostrophe
\newcommand*\apostrophe{\textsc{\char13}}

%in order to draw a text enclosed in a circle
\newcommand{\mymk}[1]{%
  \tikz \node[anchor=south west, draw,circle, inner sep=0pt, minimum size=5.5mm,
    text height=2mm]{\ensuremath{#1}} ;}
% nodes without circle
\newcommand{\mymku}[1]{%
  \tikz \node[anchor=south west, circle, inner sep=0pt, minimum size=7mm,
    text height=2mm]{\ensuremath{#1}} ;}


\counterwithin{equation}{definition}
\counterwithin{equation}{satz}