\usepackage[utf8]{inputenc}
\usepackage[T1]{fontenc}
\usepackage[hmargin={2.0cm,2.0cm},height=24cm]{geometry}
\setlength\parindent{1.1cm}
\usepackage{microtype}
%[margin=1.5cm]{geometry}
\usepackage[binary-units]{siunitx}
\usepackage[shortcuts]{extdash}
\usepackage[table,xcdraw,dvipsnames]{xcolor}%para definir colores
\usepackage[printonlyused]{acronym} %for acronyms

%\usepackage[paper=portrait,pagesize] % para poner en horizontal (landscape) la pagina

%%% Style for Todo List
%Paquete para hacer anotaciones
\usepackage{todonotes} %Paquete para hacer anotaciones
\setlength{\marginparwidth}{3.4cm}
\reversemarginpar % put the todo in the left side
\newcounter{todocounter}
\setuptodonotes{tickmarkheight=0.1cm}
%%%%%%%%%%%%%%%%%%%%%%%%%%%%%%%%%%%%%%%%%%%%%%%%%%
%%%          .----------------------> main command: \todoCodigo[]{}, 
%%%          |                                      \todoContent[]{}, 
%%%          |                                      \todoMATLAB[]{}
%%%          |
%%%          |      .---------------> 1.st Arg. (parametro adicional)
%%%          |      |        .------> 2.nd Arg. (descripción del TODO)
%%% E.g.:    |      |        |
%%%  \todoCodigo[inline]{Update Pin Remarks}
%%%%%%%%%%%%%%%%%%%%%%%%%%%%%%%%%%%%%%%%%%%%%%%%%%
%%%%%% COMANDOS %%%%%%%
%%%%%%%%%%%%%%%%%%%%%%%
%%%%%% COMANDO 01 --------------> CONTENT
\newcommand{\todoContent}[2][]
{\stepcounter{todocounter}
    \todo[
            color=caribbeangreen, 
            #1 % 1st arg. extra commands
         ]
         {  % estilo: "[ Numeration | Attribute ] Description"
            [\thetodocounter\,|\,CONTENT] #2
         } 
}
%%%%%% COMANDO 02 -------------> CODE
\newcommand{\todoCode}[2][]
{\stepcounter{todocounter}
    \todo[
            color=corn, 
            #1 % 1st arg. extra commands
         ]
         {  % estilo: "[ Numeration | Attribute ] Description"
            [\thetodocounter\,|\,CODE] #2
         }
}
%%%%%% COMANDO 03 -------------> MATLAB
\newcommand{\todoMATLAB}[2][]
{\stepcounter{todocounter}
    \todo[
            color=orange-amber, 
            #1
         ]
         {
            [\thetodocounter\,|\,MAT] 
            #2
         }
}
%%%%%%% PAQUETE EXTRA
%%%%%%% Necesario para poder poner el todo dentro de un cell en una tabla.
\usepackage{marginnote}
\let\marginpar\marginnote %enable todo in table always use \textrm{\todo{Test}}


%Constant definitions
%para listings:
%\def\SPACExLISTINGxTOxTEXT{0.2}
%vertical spaces
\def\VERTICALxSPACExLISTINGxTOxLISTING{-0.2}
\def\VERTICALxSPACExMINIPAGExTOxLISTING{0.1}
\def\VERTICALxSPACExTEXTxTOxLISTING{0.1}
\def\VERTICALxSPACExTEXTxTOxTIKZ{0.3}
\def\VERTICALxSPACExTIKZxTOxLISTING{0.15}
\def\VERTICALxSPACExITEMIZExTOxCURVE{-0.30} %usado en pedal travel
%
%horizonal spaces
\def\HORxSPACExLISTINGxTOxTIKZGRAPH{0.50}
\def\HORxSPACExTIKZGRAPHxTOxLISTING{0.20}
%

\usepackage{makeidx} % for word index
\makeindex

\usepackage{graphicx}
\graphicspath{ {figures/} }

\usepackage{lmodern}
\usepackage{lscape}
\usepackage{afterpage}
\usepackage{filecontents} %create data datatable
\usepackage{soul} %para hacer strike al texto \st{} 
%soul tambien es para : to enable highlight a una linea de codigo
\usepackage{forest} %para hacer system structure files, ver % https://tex.stackexchange.com/questions/5073/making-a-simple-directory-tree
%Tikz Settings

% for sub lists
\usepackage{tikz}
\usetikzlibrary{plotmarks}
\usetikzlibrary{shapes,arrows.meta,chains}
\usetikzlibrary {datavisualization}
\usepackage{pgfplots}
\pgfplotsset{compat=1.16}
\usepgfplotslibrary{units}
\usepgfplotslibrary{patchplots}

%para usar contadores
\usepackage{forloop}

\usepackage{stackengine}
%Para escribir codigo
\usepackage{listings,lstautogobble}[showlines=htrue,breakatwhitespace=true] 
%\usepackage{listings,lstautogobble}[showlines=htrue,breakatwhitespace=true] % first line

\usepackage{textcomp} %necessary for apostroph
%SEE: https://tex.stackexchange.com/questions/145416/how-to-have-straight-single-quotes-in-lstlistings
%  You need to load the textcomp package and add upquote=true to your \lstset command. See §4.7 of the listings documentation. Alternatively, you can simply load the upquote package, which will make all verbatim quotes single quotes

\lstloadlanguages{C, Matlab}
%%%%%%%%%%%%%%%%%%%%%%%%%
%%%%%%%%% MATLAB %%%%%%%% 
%%%%%%%%%%%%%%%%%%%%%%%%%
%some matlab definitions were extracted from: https://gist.github.com/eyliu/120689
\lstdefinestyle{Matlab-Classic}{
language=Matlab,
upquote=true, %necessary to show apostroph '
basicstyle=\fontfamily{pcr}\selectfont\footnotesize\color{black},
identifierstyle=\color{black},%
%keywordstyle=\color{darkraspberry}\bfseries, % style for keywords
keywordstyle=[1]\color{Blue}\bfseries,        % MATLAB functions bold and blue
keywordstyle=[2]\color{darkraspberry}\bfseries,,         % MATLAB function arguments purple
keywordstyle=[3]\color{Blue}\underbar,  % User functions underlined and blue
commentstyle=\usefont{T1}{pcr}{m}{sl}\color{dartmouthgreen}\small,
stringstyle=\color{Purple}, % Strings are purple
showstringspaces=false,%without this there will be a symbol in the places where there is a space
%%% Put standard MATLAB functions not included in the default
%%% language here
morekeywords={alpha,
array2table,
case,
contour,
factorial,
fillmissing,
ismissing,
movevars,
normalize,
normpdf,
normcdf,
otherwise,
poissrnd,
randi,
readtable,
rmmissing,
rng,
scatter,
smoothdata,
sortrows,
standarizeMissing,
switch,
title,
var,
writetable,
xlim,
xticks,
xticklabels,
xtickangle,
ylim,
yticks,
yticklabels,
ytickangle,
warning}
}
%%%% END MATLAB 

%%%%%%%%%%%%%%%%%%%%%%%%%
%%%%%% C-Terminal %%%%%%% 
%%%%%%%%%%%%%%%%%%%%%%%%%
\lstdefinestyle{C-Terminal}{
language=C,
basicstyle=\fontfamily{pcr}\selectfont\footnotesize\color{black},
keywordstyle=\color{darkraspberry}\bfseries, % style for keywords
upquote=true, %necessary to show apostrophe '
commentstyle=\color{dartmouthgreen}\ttfamily,
stringstyle=\color{brown(traditional)}\ttfamily, %procnamestyle=\color{brown(traditional)}\ttfamily
morecomment=*[l][\special@on\color{dartmouthgreen}\itshape]{//},
morecomment=*[s][\special@on\color{dartmouthgreen}\itshape]{/*}{*/},
%morecomment=*[n][\special@on\color{darkpastelblue}]{(0x}{x0)},
morestring=*[b][\special@on\color{byzantium}]",
%escapechar=|,%
deletekeywords={default,
double,
for,
int,
if},
otherkeywords={!,!=,~,\$,<, >=,=<,\_t,\_32},
morekeywords={char,
const,
bool,
extern,
float,
int8,
int16,
int32,
long,
uint,
uint8,
uint16,
uint32,
uint64,
unsigned,
volatile, 
void},
escapeinside={(*@}{@*)},
numbers=left, % where to put the line-numbers
breaklines=true, %para hacer wrap lines
numberstyle=\small, % the size of the fonts that are used for the line-numbers     
backgroundcolor=\color{white},
tabsize=2,
showspaces=false, % show spaces adding particular underscores
showstringspaces=false, % underline spaces within strings
showtabs=false,
alsoletter={_,\#,*},
emph={\#if, \#include, \#define, \#else, \#endif,\#ifdef,\#ifndef},
emphstyle=\color{darkraspberry},
emph={[2]
 timInfo,
 timer_alarm,
 timer_autoreload,
 timer_config,
 timer_count_dir,
 timer_group,
 timer_idx,
 timer_intr_mode,
 timer_isr,
 timer_start,
 timer_src_clk,
 gpio_num}, %end emp={[2]
emphstyle={[2]\color{darkraspberry}},
literate=  %literates come here
{./}{{{\color{red}./}}}2 %1ST LITERATE
{.^}{{{\color{red}.\^{}}}}2 {&&}{{{\color{red}\&\&{}}}}2 %{=}{{{\color{red}=}}}1 %2ND LITERATE {.}{{{\color{debianred}.}}}1 {!}{{{\color{red}!}}}1
{Ä}{{\"A}}1%
{Ö}{{\"O}}1%
{Ü}{{\"U}}1%
{ä}{{\"a}}1%
{ö}{{\"o}}1%
{ü}{{\"u}}1
{á}{{\'a}}1
{é}{{\'e}}1
{í}{{\'i}}1
{ó}{{\'o}}1
{ú}{{\'u}}1
{ñ}{{\~{n}}}1 % ñ = alt + 164
%{_}{{\_}}1
{^}{{\^{}}}1
{~}{{$\sim$}}1
{orOPER}{{||}}1
{orEqual}{{|=\,\,\,}}1
{&=}{{\&=\,\,\,}}1
%{>}{{$>$\,\,\,}}1
{<}{{$<$\,\,\,}}1
{resHochkomma}{{\textbackslash"}}1
{==}{{$==$\,\,}}1%
{equal=}{{$=$\,\,}}1%
{\%}{{\%\,\,}}1%
{e=}{{$=$\,\,\,}}1
{ß}{{\ss}}1%
{ç}{{\c{c}}}1,
framexrightmargin=5mm, 
frame=shadowbox, 
rulesepcolor=\color{bondiblue},
autogobble=true
}
%%%% END C-Terminal  

%%%%%%%%%%%%%%%%%%%%%%%%%
%%% C-classic-format %%%%
%%%%%%%%%%%%%%%%%%%%%%%%%
\lstdefinestyle{C-classic-format}{
language=C,
basicstyle=\fontfamily{pcr}\selectfont\footnotesize\color{black},
backgroundcolor=\color{white},
upquote=true, %necessary to show apostroph '
%identifierstyle=\color{black}\ttfamily,
commentstyle=\color{dartmouthgreen}\ttfamily,
stringstyle=\color{brown(traditional)}\ttfamily, %procnamestyle=\color{brown(traditional)}\ttfamily
%morecomment=[l][\color{brass}]{\#include},
%morecomment=[l][\color{brass}]{\#define},
%morecomment=[s][\color{brown(traditional)}]{/´}{´},
morecomment=*[l][\special@on\color{dartmouthgreen}\itshape]{//},
morecomment=*[s][\special@on\color{dartmouthgreen}\itshape]{/*}{*/},
morecomment=*[s][\special@on\color{darkraspberry}\ttfamily]{<}{>},
%morecomment=*[l][\special@on\color{black}\ttfamily]{ifndef},
%morecomment=*[n][\special@on\color{darkpastelblue}]{(0x}{x0)},
morestring=*[b][\special@on\color{byzantium}\bfseries]",
otherkeywords={!,!=,~,\$, >=,=<,\_t,\_32},
%escapechar=|,
%classoffset permite poner más de un color
%primero se carga el classoffset = 0, luego el classoffset = 1 y asi sucesivamente...
classoffset=0, keywordstyle=\color{darkraspberry}\bfseries, % style for keywords - SET 1
deletekeywords={ % DELETED KEYWORDS IN ORDER TO CHANGE THEM TO BLUE
bool, 
calloc,
const,
continue,
char,
double,
else, enum,
extern,
for,
float,
if,
long,
in,
int,
NULL,
pdTRUE,
return,
string,
short,
sizeof,
static, struct,
then,
typedef,
union, unsigned,
void, volatile,
while,
do  %%%%%  HERE DOORS DELETED Keywords
},  %%%%%%%%%%%%%%% END DELETED KEYWORDS
classoffset=1, %%%%%%%%%%%%% HERE COLOR IN DARKRASPBERRY
keywordstyle=\color{darkraspberry}\bfseries,
morekeywords={
int8,
int16,
int32,
List,
pdFALSE,
pdPASS,
pdTRUE,
BaseType,
Queue,
QueuePointers,
signed,
SemaphoreData,
TaskHandle,
TaskFunction,
TickType,
TimerHandle,
UBaseType,
uint,
uint8,
uint16,
uint32,
uint64,
xQUEUE}, %%%%%%%%% END KEYWORDS - SET 1  
classoffset=2, %%%%%%%%%%%%% HERE DOORS KEYWORDS
keywordstyle=\color{darkraspberry}\bfseries,
morekeywords={
AttrDef,
AttrTyp,
Baseline,
Column,
DxlObject,
Date,
ExternalLink,
Folder,
Filter,
History,
Item,
Link,
LinkRef,
Module,
ModuleVersion,
ModName_,
Skip,
Object,
Project,
View,
Sort,
real}, %%%%%%%%% END KEYWORDS - SET 1  
classoffset=5, %%%%%%%%%%%%%%% HERE KEYWORDS IN BLUE ! 
keywordstyle=\color{blue}\bfseries, %%%%% style for keywords - SET 2
morekeywords={
bool,
calloc,
char,
const,
continue,
double,
else, enum,
extern,
for,
float,
if,
in,
int,
long,
NULL,
return,
short,
string,
sizeof,
static, struct,
then,
typedef,
union, unsigned,
void, volatile,
while,
do   %% here new blue doors keywords
},   %%%%%%%%% END KEYWORDS - SET 2 
classoffset=0, %%% Finally return to normal color.
escapeinside={(*@}{@*)},
numbers=left, % where to put the line-numbers
breaklines=true, %para hacer wrap lines
numberstyle=\small, % the size of the fonts that are used for the line-numbers     
tabsize=2,
showspaces=false, % show spaces adding particular underscores
showstringspaces=false, % underline spaces within strings
showtabs=false,
alsoletter={_,\#,*},
emph={\#if, \#include, \#define, \#else, \#endif,\#ifdef,\#ifndef},
emphstyle=\color{darkraspberry},
emph={[2]
 timInfo,
 timer_alarm,
 timer_autoreload,
 timer_config,
 timer_count_dir,
 timer_group,
 timer_idx,
 timer_intr_mode,
 timer_isr,
 timer_start,
 timer_src_clk,
 gpio_num}, %end emp={[2]
emphstyle={[2]\color{darkraspberry}},
emph={[3]
    U8, U16, U32, U64,
    S8, S16, S32, S64,
    BOOLEAN,
    BITFIELD,
    DB_IBC_IN_OBJECT,
    DB_IBC_IN_WHL_OBJECT,
    DB_PROCESSING_TYPE,
    DOUBLE,
    FLOAT,
    SIZE_T,
    Os_SchedulerConfigType,
    IB_IN_VEHICLE_OBJECT,
    IB_VEHICLE_OBJECT,
    IB_VEHICLE_OBJECT_TAG,
    IB_WHEEL_OBJECT,
    T1POLESTATESCANONIC,
    T1POLESTATESDIRECT},
emphstyle={[3]\color{darkgreen}},
literate=  %literates come here
{./}{{{\color{red}./}}}2 %1ST LITERATE
{.^}{{{\color{red}.\^{}}}}2 {&&}{{{\color{red}\&\&{}}}}2 %{=}{{{\color{red}=}}}1 %2ND LITERATE {.}{{{\color{debianred}.}}}1 {!}{{{\color{red}!}}}1
{Ä}{{\"A}}1%
{Ö}{{\"O}}1%
{Ü}{{\"U}}1%
{ä}{{\"a}}1%
{ö}{{\"o}}1%
{ü}{{\"u}}1
{á}{{\'a}}1
{é}{{\'e}}1
{í}{{\'i}}1
{ó}{{\'o}}1
{ú}{{\'u}}1
{ñ}{{\~{n}}}1 % ñ = alt + 164
%{_}{{\_}}1
{^}{{\^{}}}1
{~}{{$\sim$}}1
{orOPER}{{||}}1
{orEqual}{{|=\,\,\,}}1
{&=}{{\&=\,\,\,}}1
%{>}{{$>$\,\,\,}}1
%{<}{{$<$\,\,\,}}1
{resHochkomma}{{\textbackslash"}}1
{==}{{$==$\,\,}}1%
{equal=}{{$=$\,\,}}1%
{\%}{{\%\,\,}}1%
{e=}{{$=$\,\,\,}}1
{ß}{{\ss}}1%
{ç}{{\c{c}}}1,
framexrightmargin=5mm, 
frame=shadowbox, 
rulesepcolor=\color{bondiblue},
autogobble=true
}
%

%%%% END C-classic-format 

%%%%%%%%%%%%%%%%%%%%%%%%%
%%%  Cpp-classic-format %%%%
%%%%%%%%%%%%%%%%%%%%%%%%%
\lstdefinestyle{Cpp-classic-format}{
language=C++,
basicstyle=\fontfamily{pcr}\selectfont\footnotesize\color{black},
backgroundcolor=\color{white},
upquote=true, %necessary to show apostroph '
%identifierstyle=\color{black}\ttfamily,
commentstyle=\color{dartmouthgreen}\ttfamily,
stringstyle=\color{brown(traditional)}\ttfamily, %procnamestyle=\color{brown(traditional)}\ttfamily
%morecomment=[l][\color{brass}]{\#include},
%morecomment=[l][\color{brass}]{\#define},
%morecomment=[s][\color{brown(traditional)}]{/´}{´},
morecomment=[l][\special@on\color{dartmouthgreen}\itshape]{//},
morecomment=[s][\special@on\color{dartmouthgreen}\itshape]{/*}{*/},
morecomment=*[s][\special@on\color{darkraspberry}\ttfamily]{<}{>},
%morecomment=*[l][\special@on\color{black}\ttfamily]{ifndef},
%morecomment=*[n][\special@on\color{darkpastelblue}]{(0x}{x0)},
morestring=*[b][\special@on\color{byzantium}\bfseries]",
otherkeywords={!,!=,~,\$, >=,=<,\_t,\_32,::},
%escapechar=|,
%classoffset permite poner más de un color
%primero se carga el classoffset = 0, luego el classoffset = 1 y asi sucesivamente...
classoffset=0, keywordstyle=\color{darkraspberry}\bfseries, % style for keywords - SET 1
deletekeywords={ % DELETED KEYWORDS IN ORDER TO CHANGE THEM TO BLUE
auto,
bool,
case,
calloc,
const,
continue,
char,
double,
else, enum,
extern,
for,
float,
if,
in, int,
long,
namespace,
new,
NULL,
return,
string,
short,
sizeof,
size,
\_t,
size_t,
static, struct,
switch,
std,
then,
typedef,
using,
union, unsigned,
void, volatile,
while,
do  %%%%%  HERE DOORS DELETED Keywords
},  %%%%%%%%%%%%%%% END DELETED KEYWORDS
classoffset=1, %%%%%%%%%%%%% HERE COLOR IN DARKRASPBERRY
keywordstyle=\color{darkraspberry}\bfseries,
morekeywords={
array,
int8,
int16,
int32,
List,
pdFALSE,
pdPASS,
pdTRUE,
BaseType,
Queue,
QueuePointers,
signed,
SemaphoreData,
TaskHandle,
TaskFunction,
TickType,
TimerHandle,
UBaseType,
uint,
uint8,
uint16,
uint32,
uint64,
xQUEUE}, %%%%%%%%% END KEYWORDS - SET 1  
classoffset=2, %%%%%%%%%%%%% HERE C++ 14 KEYWORDS
keywordstyle=\color{darkraspberry}\bfseries,
morekeywords={
% add here if needed
}, %%%%%%%%% END KEYWORDS - SET 1  
classoffset=5, %%%%%%%%%%%%%%% HERE ADD PREVIOUS DELETED KEYWORDS IN BLUE ! 
keywordstyle=\color{blue}\bfseries, %%%%% style for keywords - SET 2
morekeywords={
auto,
bool,
case,
calloc,
char,
char*,
const,
continue,
double,
else, enum,
extern,
for,
float,
if,
in,
int,
int*,
long,
namespace,
nanoseconds,
new,
NULL,
return,
short,
string,
size,
\_t,
size_t,
std,
sizeof,
static, struct,
switch,
then,
typedef,
using,
union, unsigned,
void, volatile,
while,
do   %% here new blue cpp keywords
},   %%%%%%%%% END KEYWORDS - SET 2 
classoffset=5, %%%%%%%%%%%%%%% HERE NEW KEYWORDS  C++ IN BLUE ! 
keywordstyle=\color{blue}\bfseries, %%%%% style for keywords - SET 3
morekeywords={
chrono,   
make_unique,
nanoseconds,
ostringstream,
unique_ptr,
vector,%% here new blue cpp keywords
},   %%%%%%%%% END KEYWORDS - SET 3 
classoffset=0, %%% Finally return to normal color.
escapeinside={(*@}{@*)},
numbers=left, % where to put the line-numbers
breaklines=true, %para hacer wrap lines
numberstyle=\small, % the size of the fonts that are used for the line-numbers     
tabsize=2,
showspaces=false, % show spaces adding particular underscores
showstringspaces=false, % underline spaces within strings
showtabs=false,
alsoletter={_,.,<,\#,*},
emph={\#if, \#include, \#define, \#else, \#endif,\#ifdef,\#ifndef},
emphstyle=\color{darkraspberry},
emph={[2]
 timInfo,
 timer_alarm,
 timer_autoreload,
 timer_config,
 timer_count_dir,
 timer_group,
 timer_idx,
 timer_intr_mode,
 timer_isr,
 timer_start,
 timer_src_clk,
 gpio_num}, %end emp={[2]
emphstyle={[2]\color{darkraspberry}},
emph={[3]
    U8, U16, U32, U64,
    S8, S16, S32, S64,
    BOOLEAN,
    BITFIELD,
    DB_IBC_IN_OBJECT,
    DB_IBC_IN_WHL_OBJECT,
    DB_PROCESSING_TYPE,
    DOUBLE,
    FLOAT,
    SIZE_T,
    Os_SchedulerConfigType,
    IB_IN_VEHICLE_OBJECT,
    IB_VEHICLE_OBJECT,
    IB_VEHICLE_OBJECT_TAG,
    IB_WHEEL_OBJECT,
    T1POLESTATESCANONIC,
    T1POLESTATESDIRECT},
emphstyle={[3]\color{darkgreen}},
literate=  %literates come here
{./}{{{\color{red}./}}}2 %1ST LITERATE
{.^}{{{\color{red}.\^{}}}}2 {&&}{{{\color{red}\&\&{}}}}2 %{=}{{{\color{red}=}}}1 %2ND LITERATE {.}{{{\color{debianred}.}}}1 {!}{{{\color{red}!}}}1
{Ä}{{\"A}}1%
{Ö}{{\"O}}1%
{Ü}{{\"U}}1%
{ä}{{\"a}}1%
{ö}{{\"o}}1%
{ü}{{\"u}}1
{á}{{\'a}}1
{é}{{\'e}}1
{í}{{\'i}}1
{ó}{{\'o}}1
{ú}{{\'u}}1
{ñ}{{\~{n}}}1 % ñ = alt + 164
%{_}{{\_}}1
{^}{{\^{}}}1
{~}{{$\sim$}}1
{orOPER}{{||}}1
{orEqual}{{|=\,\,\,}}1
{&=}{{\&=\,\,\,}}1
%{>}{{$>$\,\,\,}}1
%{<}{{$<$\,\,\,}}1
{resHochkomma}{{\textbackslash"}}1
{==}{{$==$\,\,}}1%
{equal=}{{$=$\,\,}}1%
{\%}{{\%\,\,}}1%
{e=}{{$=$\,\,\,}}1
{ß}{{\ss}}1%
{ç}{{\c{c}}}1,
framexrightmargin=5mm, 
frame=shadowbox, 
rulesepcolor=\color{bondiblue},
autogobble=true
}

%%%% END Cpp-classic-format

%Zebra option
\newcommand\realnumberstyle[1]{}

\makeatletter
\newcommand{\zebra}[3]{%
    {\realnumberstyle{#3}}%
    \begingroup
    \lst@basicstyle
    \ifodd\value{lstnumber}%
        \color{#1}%
    \else
        \color{#2}%
    \fi
        \rlap{\hspace*{\lst@numbersep}%
        \color@block{\linewidth}{\ht\strutbox}{\dp\strutbox}%
        }%
    \endgroup
}
\makeatother





%BEGIN ENVIRONMENTS
%
%%%%%%%%%%%%%%%%%%%%%%%%%%%%%%%%%%%%%%%%%%%%%%%%%%%%%%%%%%%%%%%%%%%%%%
%                             ┌──── Nr Arguments
%                             │  ┌── if 1st argument was not give then 
%                             │  │   is treated as empty
\lstnewenvironment{CPPCode}[1][]{
    \lstset{ style=Cpp-classic-format,
             frame=tlrb,
             numbers=none,
             mathescape=true,
             escapechar=¿,
             columns=flexible,
             caption=#1,    % Argument #1
             captionpos=b   % caption position
           }
}{}

%%%%%%%%%%%%%%%%%%%%%%%%%%%%%%%%%%%%%%%%%%%%%%%%%%%%%%%%%%%%%%%%%%%%%%
%escape string: escapeinside={(*@}{@*)},
\lstnewenvironment{CCode}{
    \lstset{ style=C-classic-format }
}{}

%%%%%%%%%%%%%%%%%%%%%%%%%%%%%%%%%%%%%%%%%%%%%%%%%%%%%%%%%%%%%%%%%%%%%%
%                             ┌──── Nr Arguments
%                             |  ┌── if 1st argument was not give then 
%                             |  |   is treated as empty
\lstnewenvironment{DOORSCode}[1][]{
    \lstset{ style=C-classic-format,
             frame=tlrb,
             numbers=none,
             mathescape=true,
             escapechar=¿,
             columns=flexible,
             caption=#1,    % Argument #1
             captionpos=b   % caption position
           }
}{}

%%%%%%%%%%%%%%%%%%%%%%%%%%%%%%%%%%%%%%%%%%%%%%%%%%%%%%%%%%%%%%%%%%%%%%
%                             ┌──── Nr Arguments
%                             |  ┌── if 1st argument was not give then 
%                             |  |   is treated as 1
\lstnewenvironment{DOORSCodeTypB}[1][1]{
    \lstset{ style=C-classic-format,
             frame=tlrb,
             numbers=left,
             firstnumber=#1,    % Argument #1
             mathescape=true,
             escapechar=¿,
             columns=flexible
           }
}{}

\lstnewenvironment{RTCode}{
    \lstset{ style=C-classic-format,
             frame=tlrb,
             numbers=none,
             mathescape=true,
             escapechar=¿,
             columns=flexible 
           }
}{}

\lstnewenvironment{RTCodeTypB}[1][1]{
    \lstset{ style=C-classic-format,
             frame=tlrb,
             numbers=left,
             mathescape=true,
             firstnumber=#1,
             escapechar=¿,
             columns=flexible 
           }
}{}

\lstnewenvironment{Terminal}{
    \lstset{ style=C-Terminal,
             frame=tlrb,
             numbers=none,
             mathescape=true,
             escapechar=¿,
             columns=flexible 
           }
}{}

\lstnewenvironment{MatlabCode}{
    \lstset{ style=Matlab-Classic }
}{}

\lstnewenvironment{MatlabCodeTypA}{
    \lstset{style=Matlab-Classic,frame=tlrb,numbers=none,mathescape=true,escapechar=¿,columns=flexible}
}{}

\lstnewenvironment{MatlabCodeTypOneLine}{
    \lstset{style=Matlab-Classic,frame=tlrb,numbers=none,mathescape=true,escapechar=¿,columns=flexible}
}{}
%END ENVIRONMENTS

%to Write a single line
\newcommand{\codeLine}{\begingroup \catcode`_=12 \cmdint}
    \newcommand{\cmdint}[1]{\texttt{\scantokens{#1\noexpand}}%
                \endgroup}


% NEW FUNCTIONS FOR MATLAB /SIMULINK
\newcommand{\matlabfcn}[1]{\colorbox{grayMatlab}{\texttt{#1}}}
\newcommand{\matlabvar}[1]{\texttt{#1}}
\newcommand{\matlabvalue}[1]{\texttt{#1}}
\newcommand{\matlabvarAttribute}[1]{\texttt{#1}}
\newcommand{\matlabpar}[1]{\texttt{#1}}
\newcommand{\matlabOperator}[1]{\texttt{#1}}
\newcommand{\matlabDataType}[1]{\texttt{#1}}
\newcommand{\matlabKeyword}[1]{{\color{Blue}\texttt{#1}}}

\newcommand{\simulinkBlock}[1]{{\color{Black}\texttt{#1}}}

% NEW FUNCTIONS FOR CPP
\newcommand{\CppCommonCode}[1]{{\texttt{#1}}}
\newcommand{\Cppfcn}[1]{\colorbox{grayMatlab}{\texttt{#1}}}
\newcommand{\CppKeywordCommon}[1]{{\color{blue(ryb)}\texttt{#1}}}
\newcommand{\CppKeywordSpecial}[1]{{\color{darkraspberry}\texttt{#1}}}

% NEW FUNCTIONS FOR DOORS DXL
\newcommand{\dxlMethod}[1]{{\texttt{#1}}}
\newcommand{\dxlKeywordCommon}[1]{{\color{blue(ryb)}\texttt{#1}}}
\newcommand{\dxlKeywordSpecial}[1]{{\color{darkraspberry}\texttt{#1}}}
\newcommand{\dxlString}[1]{{\color{brown(traditional)}\texttt{"#1"}}}
\newcommand{\dxlStringDeclAssig}[2]{\texttt{#1}\,$=$\,{\color{brown(traditional)}\texttt{"#2"}}}
\newcommand{\dxlStringDeclAssign}[2]{\codeLine{#1}\,$=$\,{\color{brown(traditional)}\texttt{"#2"}}}
\newcommand{\dxlDeclAssign}[2]{\codeLine{#1}\,$=$\,\texttt{"#2"}}
\newcommand{\dxlDeclAssignMethod}[2]{\codeLine{#1}\,$=$\,\codeLine{#2}}
\newcommand{\dxlObjectAssign}[3]{\codeLine{#1}$\rightarrow${\color{brown(traditional)}\texttt{"#2"}}  \,$=$\,{\color{brown(traditional)}\texttt{"#3"}}}
\newcommand{\dxlObjectAssignCode}[3]{\codeLine{#1}$\rightarrow${\color{brown(traditional)}\texttt{"#2"}}  \,$=$\,{\codeLine{#3}}}


%COMMENT TEXT
\newcommand{\codecomment}[1]{\color{dartmouthgreen}{\ttfamily{#1}}}
\newcommand{\codeOper}[1]{\color{blue}{\texttt{#1}}}
%NEW LINE (code green plus) 
\newcommand{\cgp}{\color{dartmouthgreen}{\texttt{+}}}
%DELETED LINE (code red minus) 
\newcommand{\crm}{\color{debianred}{\texttt{-}}}



%interesting url
% -Customizing a code listing environment: https://latex.org/forum/viewtopic.php?t=102

%if you are looking for definitions like
% \MPWxSMALLxLISTING{0.6} 
% refer to: styleDefsConstants.tex 

% \lstset{ %language=[Sharp]C, %necesario para el emphasize,
% }
\usepackage{filecontents} %create data datatable


%for recognize the comment colors
% Style to select only points from #1 to #2 (inclusive)
\makeatletter

\newif\ifspecial@env@
\def\special@on{\global\special@env@true}
\def\special@off{\global\special@env@false}

\lst@AddToHook{DetectKeywords}{%
    \global\let\last@lst@thestyle=\lst@thestyle
}

\def\emphstyle{%
    \last@lst@thestyle
    \aftergroup\special@off
    \underbar
}

\def\keywordstyle{%
    \ifspecial@env@
        \last@lst@thestyle
    \else
        \color{blue}%
    \fi
    \aftergroup\special@off
}
\makeatother


%extracted from:
%https://tex.stackexchange.com/questions/7032/good-way-to-make-textcircled-numbers
\newcommand*\circled[1]{\tikz[baseline=(char.base)]{
            \node[shape=circle,draw,inner sep=2pt] (char) {#1};}}
% \newcommand*\circledMath[1]{\tikz[baseline=(char.base)]{
%             \node[shape=circle,draw,inner sep=2pt] (char) {${#1}$};}}

%Mathematical Packages
\usepackage{textcomp} %para tener el simbolo de division en texto ver: https://tex.stackexchange.com/questions/108035/is-there-a-way-to-produce-this-division-symbol-%C3%B7
\usepackage{enumerate} %para poder enumerar con romanos (i),..
\usepackage{etoolbox}
\usepackage{amsfonts} %for R or C font
\usepackage{mathtools, amssymb}
\newtagform{bold}{\bfseries(}{)}
\usetagform{bold}
\usetikzlibrary{calc} % for ticks plots
\usepackage[thref, amsmath, thmmarks]{ntheorem}%

\usepackage{chngcntr} %para contador

    
\usepackage{stackengine}
\newcommand\exponvar[2]
{
    \stackMath\Shortstack[c]{{\scalebox{.7}{#2}} {#1}}
}
\newcommand\TranslatedText[2]
{
    \stackengine{\stackgap}{\text{#1}}{{\scalebox{.7}{(#2)}}}{U}{\stackalignment}{\quietstack}{\useanchorwidth}{\stacktype}\!\!
}

\usepackage{physics}

%References packages
%ALWAYS FIRST CAPTION THEN HYPERREF
\usepackage{caption} 
\usepackage{hyperref} %para url
\hypersetup{
    colorlinks=true,
    linkcolor=blue,
    filecolor=magenta,      
    urlcolor=brandeisblue,
}

\usepackage{cleveref} %for references

\newcommand\note[1][]{\subsection{#1}\ifblank{#1}{\hskip-\labelsep}{\relax}}% 


%para tablas
\usepackage{colortbl}
\usepackage{multicol}
\usepackage{multirow}
\usepackage{makecell}
\usepackage{hhline}
\usepackage{array}
\usepackage{tabularray}
\UseTblrLibrary{counter} %in order to fix error of counters!
%for good fracs numbers visualization in tables
\newcolumntype{C}{>{$}c<{$}}
\newlength\llength
\llength=1.38ex\relax
% Math commands
% usDef = user Defined
\makeatletter
\newtheoremstyle{myplain}%
{\item[\hskip\labelsep \theorem@headerfont ##1\ \textup{\bfseries(##2)}\ \,\theorem@separator]}%
{\item[\hskip\labelsep \theorem@headerfont ##1\ \textup{\bfseries(##2)}\ (##3)\,\theorem@separator]}
\makeatother
\theoremstyle{myplain}
\theoremheaderfont{\itshape}
\theorembodyfont{\itshape}
\theoremseparator{\textemdash}
\newtheorem{definition}{Definition}[section]
\newtheorem{satz}{Satz}[section]
\newcommand{\vspaceBemerkung}{\vspace{0.2cm}}
\newcommand{\entspricht}{{\,\hat{=}\,}}
\renewcommand{\qed}{\hfill\blacksquare}
\newcommand{\qedwhite}{\hfill \ensuremath{\Box}}
\newcommand{\mbb}[1]{\mathbb{R}^{#1}}
\newcommand{\mbbb}[2]{\mathbb{R}^{#1\times{#2}}}
\newcommand*\conj[1]{\bar{#1}}
\newcommand*\mean[1]{\bar{#1}}
\newcommand\widebar[1]{\mathop{\overline{#1}}} 

\DeclareMathOperator{\Lagr}{\mathcal{L}}
\DeclareMathOperator{\Lapl}{\mathcal{L}}

%new command for apostrophe
\newcommand*\apostrophe{\textsc{\char13}}

%in order to draw a text enclosed in a circle
\newcommand{\mymk}[1]{%
  \tikz \node[anchor=south west, draw,circle, inner sep=0pt, minimum size=5.5mm,
    text height=2mm]{\ensuremath{#1}} ;}
% nodes without circle
\newcommand{\mymku}[1]{%
  \tikz \node[anchor=south west, circle, inner sep=0pt, minimum size=7mm,
    text height=2mm]{\ensuremath{#1}} ;}


\counterwithin{equation}{definition}
\counterwithin{equation}{satz}
\newcommand{\camouflagegreenCellColor}{\cellcolor[rgb]{0.47, 0.53, 0.42}}
\newcommand{\lavenderCellColor}{\cellcolor[rgb]{0.9, 0.9, 0.98}}
\newcommand{\TabelleArrayColorGray}{\arrayrulecolor[rgb]{0.812,0.812,0.812}}
\newcommand{\TabelleRowColorGray}{\rowcolor[rgb]{0.812,0.812,0.812}}
\newcommand{\TabelleRowCellGray}{\cellcolor[rgb]{0.812,0.812,0.812}}
% MicroController Commands
% usDef = User Defined

\newcommand{\BASEPRI}{\texttt{BASEPRI }}
\newcommand{\FIFO}{\texttt{FIFO }}
\newcommand{\IO}{\texttt{I/O }}
\newcommand{\idle}{\texttt{idle }}
\newcommand{\ISR}{\texttt{ISR }}
\newcommand{\IPSR}{\texttt{IPSR }}
\newcommand{\IRQ}{\texttt{IRQ }}
\newcommand{\ICR}{\texttt{ICR }}
\newcommand{\mailbox}{\texttt{mailbox }}
\newcommand{\NVIC}{\texttt{NVIC }}
\newcommand{\PRIMASK}{\texttt{PRIMASK }}
\newcommand{\RAM}{\texttt{RAM }}
\newcommand{\register}{\texttt{register }}
\newcommand{\ROM}{\texttt{ROM }}
\newcommand{\SysTick}{\texttt{SysTick }}
\newcommand{\TExaS}{\texttt{TExaS }}
\newcommand{\UART}{\texttt{UART }}

%new custom commands, defined by Kike.
\newcommand{\ADCbit}[1]{\texttt{ADC{#1}}}
\newcommand{\ADCREG}[1]{\texttt{ADC\_{#1}\_R}}
\newcommand{\bitsRange}[2][50]{\texttt{{#1}-{#2}}}
\newcommand{\CustomHex}[2][0000]{\texttt{0x{#1}.{#2}}}
\newcommand{\GPIOPort}[1]{\texttt{GPIO\_PORT{#1}}}
\newcommand{\GPIOPortR}[2][A]{\texttt{GPIO\_PORT{#1}\_{#2}\_R}}
\newcommand{\GPIOPortHandler}[1]{\texttt{GPIO\_PORT{#1}\_Handler}}
\newcommand{\HandlerISR}[1]{\texttt{#1\_Handler}}
\newcommand{\IRQnr}[1]{\texttt{{#1}}}
\newcommand{\NVICPRI}[1]{\texttt{NVIC\_PRI{#1}\_R}}
\newcommand{\NVICEN}[1]{\texttt{NVIC\_EN{#1}\_R}}
\newcommand{\NVICDIS}[1]{\texttt{NVIC\_DIS{#1}\_R}}
\newcommand{\NVICST}[1]{\texttt{NVIC\_ST\_{#1}\_R}}
\newcommand{\Ttimer}[2][A]{\texttt{Timer\_{#2}{#1}}}
\newcommand{\xNrbit}[1]{$#1$-\texttt{bit}}
\newcommand{\xNrbits}[1]{$#1$-\texttt{bits}}
\newcommand{\volties}[2][0]{$\si{{#1}\volt}_{#2}$}
\newcommand{\volti}[1]{$\si{{#1}\volt}$}
\newcommand{\voltiposi}[1]{$+\si{{#1}\volt}$}
\newcommand{\voltinega}[1]{$-\si{{#1}\volt}$}

%To easy write double backslash
\newcommand{\dueBackslash}{\textbackslash\textbackslash}

% \renewcommand{\labelenumii}{\arabic{enumi}.\arabic{enumii}}
% \renewcommand{\labelenumiii}{\arabic{enumi}.\arabic{enumii}.\arabic{enumiii}}
% \renewcommand{\labelenumiv}{\arabic{enumi}.\arabic{enumii}.\arabic{enumiii}.\arabic{enumiv}}
%\renewcommand{\labelenumii}{\theenumii}
%\renewcommand{\theenumii}{\theenumi.\arabic{enumii}.}