%color definitions
% \definecolor{ashgrey}{rgb}{0.7, 0.75, 0.71}
% \definecolor{bondiblue}{rgb}{0.0, 0.58, 0.71}
% \definecolor{blue(ryb)}{rgb}{0.01, 0.28, 1.0}
% \definecolor{byzantium}{rgb}{0.44, 0.16, 0.39} %morado
% \definecolor{darkgreen}{rgb}{0.0, 0.2, 0.13}
% \definecolor{darkspringgreen}{rgb}{0.09, 0.45, 0.27}
% \definecolor{dartmouthgreen}{rgb}{0.05, 0.5, 0.06}
% \definecolor{debianred}{rgb}{0.84, 0.04, 0.33}
% \definecolor{mygray}{rgb}{0.5,0.5,0.5}
% \definecolor{aurometalsaurus}{rgb}{0.43, 0.5, 0.5}
% \definecolor{asparagus}{rgb}{0.53, 0.66, 0.42}
% \definecolor{arylideyellow}{rgb}{0.91, 0.84, 0.42}
% \definecolor{brown(traditional)}{rgb}{0.59, 0.29, 0.0}
% \definecolor{brass}{rgb}{0.71, 0.65, 0.26}
% \definecolor{carrotorange}{rgb}{0.93, 0.57, 0.13}
% \definecolor{darkpastelblue}{rgb}{0.47, 0.62, 0.8}
% \definecolor{brandeisblue}{rgb}{0.0, 0.44, 1.0}

\usepackage[utf8]{inputenc}
\usepackage[binary-units]{siunitx}
\usepackage[shortcuts]{extdash}
\usepackage[margin=1.5cm]{geometry}
\usepackage[table,xcdraw]{xcolor}
\usepackage[printonlyused]{acronym} %for acronyms
\usepackage{hyperref} %para url
\hypersetup{
    colorlinks=true,
    linkcolor=blue,
    filecolor=magenta,      
    urlcolor=brandeisblue,
}
\usepackage{mathtools} % loads amsmath
\usepackage{textcomp} %\textregistered % or \textcopyright
\usepackage{soul} % para tachar texto con una linea e.g \st{texto tachado}
\usepackage{colortbl}
\usepackage{multicol}
\usepackage{multirow}
\usepackage{makecell}
\usepackage{hhline}
\usepackage{array}
\usepackage{graphicx}
\usepackage{caption}
\usepackage{lscape}
\usepackage{afterpage}
\usepackage{pgf}
\usepackage{cleveref} %for references
%music packages
\usepackage{musicography}
\usepackage{stackengine}
%%para escribir codigo en LaTeX
\usepackage{soul} %% to enable highlight a una linea de codigo
\usepackage{listings,lstautogobble}[showlines=htrue,breakatwhitespace=true] 
\lstloadlanguages{C++}
\lstset{ %language=[11]C, %necesario para el emphasize,
basicstyle=\fontfamily{pcr}\selectfont\footnotesize\color{black},
keywordstyle=\color{blue(ryb)}\bfseries, % style for keywords
commentstyle=\color{dartmouthgreen}\ttfamily,
stringstyle=\color{brown(traditional)}\ttfamily, %procnamestyle=\color{brown(traditional)}\ttfamily
%morecomment=[l][\color{brass}]{\#include},
%morecomment=[l][\color{brass}]{\#define},
%morecomment=[s][\color{brown(traditional)}]{/´}{´},
morecomment=*[l][\special@on\color{dartmouthgreen}\itshape]{//},
morecomment=*[s][\special@on\color{dartmouthgreen}\itshape]{/*}{*/},
%morecomment=*[n][\special@on\color{darkpastelblue}]{(0x}{x0)},
morestring=*[b][\special@on\color{byzantium}]",
%escapechar=|,%
otherkeywords={!,!=,~,\$,>=,=<,\_t,\_32},
morekeywords={char,
class,
const,
bool,
double,
explicit,
float,
int,
int8,
int16,
int32,
long,
List,
public,
private,
sizeof,
short,
static,
struct,
typedef,
union,
BaseType,
Queue,
QueuePointers,
SemaphoreData,
TaskHandle,
TaskFunction,
TickType,
TimerHandle,
UBaseType,
uint,
uint8,
uint16,
uint32,
uint64,
unsigned,
volatile, 
void,
xQUEUE},
escapeinside={(*@}{@*)},
numbers=left, % where to put the line-numbers
breaklines=true, %para hacer wrap lines
numberstyle=\small, % the size of the fonts that are used for the line-numbers     
backgroundcolor=\color{white},
tabsize=2,
showspaces=false, % show spaces adding particular underscores
showstringspaces=false, % underline spaces within strings
showtabs=false,
alsoletter={_,\#,*},
emph={\#if, \#include, \#define, \#else, \#endif,\#ifdef,\#ifndef},
emphstyle={[2]\color{blue}},
emphstyle=\color{darkscarlet},
 emph={[2]
 timInfo,
 timer_alarm,
 timer_autoreload,
 timer_config,
 timer_count_dir,
 timer_group,
 timer_idx,
 timer_intr_mode,
 timer_isr,
 timer_start,
 timer_src_clk,
 gpio_num
},
literate={./}{
{{\color{red}./}}}2 
{.^}{{{\color{red}.\^{}}}}2 {&&}{{{\color{red}\&\&{}}}}2 %{=}{{{\color{red}=}}}1 {.}{{{\color{debianred}.}}}1 {!}{{{\color{red}!}}}1
{Ä}{{\"A}}1%
{Ö}{{\"O}}1%
{Ü}{{\"U}}1%
{ä}{{\"a}}1%
{ö}{{\"o}}1%
{ü}{{\"u}}1
{á}{{\'a}}1
{é}{{\'e}}1
{í}{{\'i}}1
{ó}{{\'o}}1
{ú}{{\'u}}1
{ñ}{{\~{n}}}1 % ñ = alt + 164
%{_}{{\_}}1
{^}{{\^{}}}1
{~}{{$\sim$}}1
{orOPER}{{||}}1
{orEqual}{{|=\,\,\,}}1
{&=}{{\&=\,\,\,}}1
{>}{{\,\texttt{>}\,}}1
{<}{{\texttt{<}\,}}1
{'}{{\,\texttt{\textquotesingle}\,}}1
{resHochkomma}{{\textbackslash"}}1
{==}{{$==$\,\,}}1%
{equal=}{{$=$\,\,}}1%
{\%}{{\%\,\,}}1%
{e=}{{$=$\,\,\,}}1
{ß}{{\ss}}1%
{ç}{{\c{c}}}1,
framexrightmargin=5mm, 
frame=shadowbox, 
rulesepcolor=\color{bondiblue},
autogobble=true}
\usepackage{filecontents} %create data datatable
\usepackage{pgfplots}
\pgfplotsset{compat=1.16}
\usepgfplotslibrary{units}
\usepackage{forloop}
%\usepackage{fmtcount}

%for good fracs numbers visualization in tables
\newcolumntype{C}{>{$}c<{$}}



% normal line of code
% highlighted line of code
% lighter blue highlight
% darker blue highlight 

%%%%%%%%%%%%%%%%%%%%%%%%%%%%%%%%%%%%%%%%%%%%%%%%%%%%%%%%%%%%%%%%%%%%%%%%%%%%%%

\newcommand\realnumberstyle[1]{}

\makeatletter
\newcommand{\zebra}[3]{%
    {\realnumberstyle{#3}}%
    \begingroup
    \lst@basicstyle
    \ifodd\value{lstnumber}%
        \color{#1}%
    \else
        \color{#2}%
    \fi
        \rlap{\hspace*{\lst@numbersep}%
        \color@block{\linewidth}{\ht\strutbox}{\dp\strutbox}%
        }%
    \endgroup
}
\makeatother

%Tikz package
\usepackage{tikz}
\usepackage{verbatim}




% Style to select only points from #1 to #2 (inclusive)
\makeatletter

\newif\ifspecial@env@
\def\special@on{\global\special@env@true}
\def\special@off{\global\special@env@false}

\lst@AddToHook{DetectKeywords}{%
    \global\let\last@lst@thestyle=\lst@thestyle
}

\def\emphstyle{%
    \last@lst@thestyle
    \aftergroup\special@off
    \underbar
}

\def\keywordstyle{%
    \ifspecial@env@
        \last@lst@thestyle
    \else
        \color{blue}%
    \fi
    \aftergroup\special@off
}
\makeatother


%\usepackage[active]{preview}
%\PreviewEnvironment{tikzpicture}
\usetikzlibrary{shapes,arrows,chains}
%\setlength\PreviewBorder{5mm}%
%FLOW CHART COMMAND

%pagina en blanco
\newcommand\blankpage{%
    \null
    \thispagestyle{empty}%
    \addtocounter{page}{-1}%
    \newpage}
    
%% HERE SOME TEXT SCPAPES
\newcommand{\BASEPRI}{\texttt{BASEPRI }}
\newcommand{\FIFO}{\texttt{FIFO }}
\newcommand{\IO}{\texttt{I/O }}
\newcommand{\idle}{\texttt{idle }}
\newcommand{\ISR}{\texttt{ISR }}
\newcommand{\IPSR}{\texttt{IPSR }}
\newcommand{\IRQ}{\texttt{IRQ }}
\newcommand{\ICR}{\texttt{ICR }}
\newcommand{\mailbox}{\texttt{mailbox }}
\newcommand{\NVIC}{\texttt{NVIC }}
\newcommand{\PRIMASK}{\texttt{PRIMASK }}
\newcommand{\RAM}{\texttt{RAM }}
\newcommand{\register}{\texttt{register }}
\newcommand{\ROM}{\texttt{ROM }}
\newcommand{\SysTick}{\texttt{SysTick }}
\newcommand{\TExaS}{\texttt{TExaS }}
\newcommand{\UART}{\texttt{UART }}



%new custom commads, defined by Kike.
\newcommand{\ADCbit}[1]{\texttt{ADC{#1}}}
\newcommand{\ADCREG}[1]{\texttt{ADC\_{#1}\_R}}
\newcommand{\bitsRange}[2][50]{\texttt{{#1}-{#2}}}
\newcommand{\CustomHex}[2][0000]{\texttt{0x{#1}.{#2}}}
\newcommand{\GPIOPort}[1]{\texttt{GPIO\_PORT{#1}}}
\newcommand{\GPIOPortR}[2][A]{\texttt{GPIO\_PORT{#1}\_{#2}\_R}}
\newcommand{\GPIOPortHandler}[1]{\texttt{GPIO\_PORT{#1}\_Handler}}
\newcommand{\HandlerISR}[1]{\texttt{#1\_Handler}}
\newcommand{\IRQnr}[1]{\texttt{{#1}}}
\newcommand{\NVICPRI}[1]{\texttt{NVIC\_PRI{#1}\_R}}
\newcommand{\NVICEN}[1]{\texttt{NVIC\_EN{#1}\_R}}
\newcommand{\NVICDIS}[1]{\texttt{NVIC\_DIS{#1}\_R}}
\newcommand{\NVICST}[1]{\texttt{NVIC\_ST\_{#1}\_R}}
\newcommand{\Ttimer}[2][A]{\texttt{Timer\_{#2}{#1}}}
\newcommand{\xNrbit}[1]{$#1$-\texttt{bit}}
\newcommand{\xNrbits}[1]{$#1$-\texttt{bits}}
\newcommand{\camouflagegreenCellColor}{\cellcolor[rgb]{0.47, 0.53, 0.42}}
\newcommand{\lavenderCellColor}{\cellcolor[rgb]{0.9, 0.9, 0.98}}
\newcommand{\TabelleRowColorGray}{\rowcolor[rgb]{0.812,0.812,0.812}}
\newcommand{\TabelleArrayColorGray}{\arrayrulecolor[rgb]{0.812,0.812,0.812}}
\newcommand{\TabelleRowCellGray}{\cellcolor[rgb]{0.812,0.812,0.812}}
\newcommand{\volties}[2][0]{$\si{{#1}\volt}_{#2}$}
\newcommand{\volti}[1]{$\si{{#1}\volt}$}
\newcommand{\voltiposi}[1]{$+\si{{#1}\volt}$}
\newcommand{\voltinega}[1]{$-\si{{#1}\volt}$}



%flow chart color definitions
\definecolor{ballblue}{rgb}{0.13, 0.67, 0.8}
\definecolor{celadon}{rgb}{0.67, 0.88, 0.69}
\definecolor{coralred}{rgb}{1.0, 0.25, 0.25}

\colorlet{lcfree}{celadon}
\colorlet{lcnorm}{ballblue}
\colorlet{lccong}{coralred}