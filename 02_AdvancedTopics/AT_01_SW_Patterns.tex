\subsection{Software Patterns}
\label{subsec:AT-01-SW-Patterns}
%~\cref{subsec:AT-01-SW-Patterns}

A software pattern describes a solution to a common problem arising within a context. There are many context where the software pattern arrives, e.g. mobile phones, air-crafts, and so on. As you see is easy to understand that in many fields there are a recurrent solution within a certain context in order to solve a recurrent design problem.\\

Patterns helps improve software quality and developer productivity by
\begin{itemize}
    \item \textbf{Naming} recurring design structures
    \item \textbf{Specifying} design structure explicitly by identifying key properties of classes and objects, e.g. 
        \begin{itemize}
            \item roles \& relationships
            \item dependecies
            \item interactions
            \item conventions
        \end{itemize}
    \item \textbf{Abstracting} from concrete design elements
    \item \textbf{Distilling and codifying knowledge} gleaned from the successful design experience of experts.
\end{itemize}

Patterns describe both a \textbf{thing} and a \textbf{process}.

\begin{itemize}
    \item The "thing" (the "what" ) typically means a particular high-level design outline or description of implementation details.

    \item The "process" (the "how") typically describes the steps to perform to create a thing.
\end{itemize}